\documentclass[aspectratio=169]{beamer}

\usepackage[utf8]{inputenc}
\usepackage[T1]{fontenc}
\usepackage[french]{babel}

\usepackage{./pkg/commonstyles}
\usepackage{./pkg/presstyle}

\usepackage{vdr}	

\usepackage{pgfplots}
\usepgfplotslibrary{groupplots}
\usetikzlibrary{backgrounds}
\usetikzlibrary{positioning}
\usepackage[list-style=itemize, backref=true, mark-format={ }]{enotez}

\usetikzlibrary{external}
\tikzsetexternalprefix{pdffig/}
%\tikzexternalize

\title{VDR-4 -- Imagier}
\institute{}
\titlegraphic{}

\begin{document}

\maketitle

\begin{frame}{Cycle à haute et basse fréquence}
	\pgfplotsset{lfhf/.style={
			height = 0.42 \textheight,
			enlarge y limits = {value=0.9, upper},
			ylabel=Pression (hPa),
}}

\newcommand{\istart}{2}
\newcommand{\tic}{2}
\begin{tikzpicture}
	\begin{axis}[lfhf, xlabel=]

		\addplot +[]table[x=time, y=Pao] {dat/simvent1.dat};
		\draw [plage](axis cs:\istart,45) -- (axis cs:\istart + \tic, 45) node[midway, above] {Inspi.};
		\draw [plage](axis cs:\istart + \tic,45) -- (axis cs:\istart + 2*\tic, 45) node[midway, above] {Expi.};

		\draw [dashed] 
		(axis cs: \istart,\pgfkeysvalueof{/pgfplots/ymax}) -- (axis cs:\istart,0)
		(axis cs: \istart + \tic,\pgfkeysvalueof{/pgfplots/ymax}) -- (axis cs:\istart + \tic,0)
		(axis cs: \istart + 2 *\tic,\pgfkeysvalueof{/pgfplots/ymax}) -- (axis cs:\istart + 2*\tic,0);
	\end{axis}

\end{tikzpicture}
\vskip 0.25em

\newcommand{\pstart}{2.085}
\newcommand{\tip}{0.059}

\begin{tikzpicture}
\begin{axis}[lfhf, xlabel=Temps (s)]
	\addplot +[restrict x to domain=1:3]table[x=time, y=Pao] {dat/simvent1.dat};
	\draw [dashed] 
		(axis cs: \pstart,\pgfkeysvalueof{/pgfplots/ymax}) -- (axis cs:\pstart,0)
		(axis cs: \pstart + \tip,\pgfkeysvalueof{/pgfplots/ymax}) -- (axis cs:\pstart + \tip,0)
		(axis cs: \pstart + 2 *\tip,\pgfkeysvalueof{/pgfplots/ymax}) -- (axis cs:\pstart + 2*\tip,0);
		\draw [plage] (axis cs:\pstart,45) -- (axis cs:\pstart + \tip, 45) node[midway, above] {i};
		\draw [plage](axis cs:\pstart + \tip,45) -- (axis cs:\pstart + 2*\tip, 45) node[midway, above] {e};

\end{axis}
\end{tikzpicture}

\end{frame}

\begin{frame}{Fonctionnement du phasitron}
	\input{fig/fig-phasitron-coupe}
\end{frame}

\begin{frame}{Pressions moyennes}
	\input{fig/fig-multimeter}
\end{frame}

\begin{frame}{Pression motrice}
	\newcommand{\pexp}{5}
\newcommand{\pins}{18}
\newcommand{\arrpos}{1}

\begin{tikzpicture}
	\begin{axis}[]

		\addplot +[restrict x to domain=0:5]table[x=time, y=Pao] {dat/simvent1.dat};


		\draw [dashed] 
		(axis cs: \pgfkeysvalueof{/pgfplots/xmin},\pexp) -- (axis cs: \pgfkeysvalueof{/pgfplots/xmax},\pexp)
		(axis cs: \pgfkeysvalueof{/pgfplots/xmin},\pins) -- (axis cs: \pgfkeysvalueof{/pgfplots/xmax},\pins);

		\draw [->](axis cs:\arrpos,\pexp) -- (axis cs:\arrpos, \pins) node [midway, left]{$P_{motrice}$};
	\end{axis}

\end{tikzpicture}

\end{frame}

\begin{frame}{Ratio I:E normal et inversé}
	\def\iehuit{%
\addplot graphics [
	xmin=0,
	ymin=0,
	xmax=1,
	ymax=60
]}

\begin{tikzpicture}

\begin{groupplot}[
group style={
	group size=1 by 2,
	xlabels at=edge bottom
},
enlargelimits=false,
height=0.46\textheight,
]

\nextgroupplot
\iehuit {img/509.jpg};

\nextgroupplot
\iehuit{img/828.jpg};

\end{groupplot}
\end{tikzpicture}

\end{frame}

\begin{frame}{Ratio I:E normal et inversé}
	\def\iehuit{%
\addplot graphics [
	xmin=0,
	ymin=0,
	xmax=8,
	ymax=60
]}

\begin{tikzpicture}

\begin{groupplot}[
group style={
	group size=1 by 2,
	xlabels at=edge bottom
},
enlargelimits=false,
height=0.46\textheight,
]

\nextgroupplot
\iehuit {img/329.jpg};

\nextgroupplot
\iehuit{img/629.jpg};

\end{groupplot}
\end{tikzpicture}

\end{frame}

\begin{frame}{Augmentation des résistances}
	\centering
	\begin{tikzpicture}
		\begin{groupplot}[
				group style={
					group size=2 by 1,
					ylabels at=edge left
				},
				width=0.48\textwidth,
				restrict x to domain=1.5:4,
				every axis plot post/.style={
					mark=none
				},
				xlabel=Temps (s),
				ylabel=Pression (hPa),
				enlargelimits=false,
				ymax=40
			]
			\nextgroupplot[title={Raw = 5 hPa/l/s}]
			\addplot [color=marinechum] table[x=time, y=Pao, color=marinechum] {dat/raw5.dat};

		 \nextgroupplot[title={Raw = 15 hPa/l/s}]
			\addplot [color=marinechum] table[x=time, y=Pao, color=marinechum] {dat/raw15.dat};
		\end{groupplot}
	\end{tikzpicture}

\end{frame}

\begin{frame}{Augmentation des résistances}
	\centering
	\begin{tikzpicture}
		\begin{groupplot}[
				group style={
					group size=2 by 1,
					ylabels at=edge left
				},
				width=0.48\textwidth,
				restrict x to domain=1.5:4,
				every axis plot post/.style={
					mark=none
				},
				xlabel=Temps (s),
				ylabel=Pression (hPa),
				enlargelimits=false,
				ymax=45,
			]
			\nextgroupplot[title={Raw = 6 hPa/l/s}]
			\addplot [color=marinechum] table[x=time, y=Pao, color=marinechum] {dat/raw6.dat};

			\nextgroupplot[title={Raw = 9 hPa/l/s}]
			\addplot [color=marinechum] table[x=time, y=Pao, color=marinechum] {dat/raw9.dat};
		\end{groupplot}
	\end{tikzpicture}

\end{frame}

\begin{frame}{Cartouche pneumatique}
	\input{fig/fig-cartouche}
\end{frame}

\begin{frame}[b]
	%\frametitle{Circuit logique}
	\centering
	\includegraphics[height=\textheight]{img/circuit-logique.pdf}
	\note {Pas si logique que ça}
\end{frame}

\begin{frame}{Paramètres d'amplitude}
	\input{fig/fig-interaction}
\end{frame}

\begin{frame}[c]{Paramètre de cyclage haute fréquence}
	\centering
	\tikzset{external/remake next}
	\input{fig/fig-controleie}
\end{frame}

\begin{frame}[c]{Séquence des ajustements}
	\centering
	\input{fig/fig-sequence}
\end{frame}

\end{document}
